\chapter{Summary of the results}

\section{Top 10 results using vector models, with and without idf terms weighting}

\begin{table}[h!]
\begin{center}
\begin{tabular}{ | c c c c c c c c c | }
\hline
\textbf{Pos} & \textbf{Case1} & \textbf{Case2} & \textbf{Case3} & \textbf{Case4} & \textbf{Case5} & \textbf{Case6} & \textbf{Case7} & \textbf{Case8} \\ \hline
1 & t3.1 & t10.2 & t1.10 & t8.3 & t12.10 & t1.3 & t20.2 & t11.3 \\
2 & t10.2 & t19.5 & t1.14 & t19.6 & t1.6 & t11.3 & t21.1 & t10.11 \\
3 & t23.1 & t10.1 & t19.1 & t24 & t12.9 & t16.9 & t22.4 & t1.3 \\
4 & t14.5 & t10.3 & t3.1 & t8.4 & t19.6 & t1.11 & t20.1 & t1.10 \\
5 & t15.1 & t24.1 & t7.9 & t8.1 & t17.2 & t10.2 & t19.2 & t5.1 \\
6 & t21.1 & t11.1 & t8.1 & t15.3 & t13.2 & t10.3 & t17.3 & t1.13 \\
7 & t19.3 & t10.8 & t1.2 & t10.2 & t22.1 & t11.4 & t14.1 & t1.12 \\
8 & t8.1 & t22.3 & t1.11 & t19.5 & t12.3 & t1.10 & t6.2 & t1.9 \\
9 & t19.2 & t8.3 & t10.3 & t12.8 & t5.5 & t1.7 & t6.5 & t11.4 \\
10 & t16.13 & t14.5 & t8.8 & t8.6 & t1.17 & t12.3 & t22.2 & t16.5 \\
\hline
\end{tabular}
\end{center}
\caption{Results vector space model(with idf weighting)}
\end{table}

\begin{table}[h!]
\begin{center}
\begin{tabular}{ | c c c c c c c c c | }
\hline
\textbf{Pos} & \textbf{Case1} & \textbf{Case2} & \textbf{Case3} & \textbf{Case4} & \textbf{Case5} & \textbf{Case6} & \textbf{Case7} & \textbf{Case8} \\ \hline
1 & t3.1 & t3.3 & t1.10 & t8.3 & t5.5 & t2.2 & t20.2 & t11.3 \\
2 & t15.1 & t8.9 & t15.1 & t2.2 & t1.10 & t2.1 & t20.1 & t1.7 \\
3 & t16.13 & t10.2 & t2.1 & t12.3 & t17.2 & t8.3 & t21.1 & t1.16 \\
4 & t14.1 & t1.7 & t6.2 & t8.10 & t8.9 & t12.3 & t22.4 & t24.1 \\
5 & t14.3 & t2.2 & t8.3 & t6.5 & t6.5 & t8.10 & t22.2 & t1.10 \\
6 & t8.1 & t2.1 & t1.7 & t5.4 & t10.2 & t5.4 & t19.2 & t6.2 \\
7 & t24.2 & t6.5 & t16.7 & t8.9 & t12.3 & t12.6 & t5.5 & t8.9 \\
8 & t10.2 & t12.3 & t4.6 & t1.10 & t1.7 & t17.1 & t12.1 & t19.6 \\
9 & t16.8 & t5.4 & t3.1 & t2.1 & t3.1 & t3.3 & t6.2 & t10.2 \\
10 & t7.7 & t8.3 & t21.1 & t10.2 & t2.2 & t8.9 & t17.3 & t3.3 \\
\hline
\end{tabular}
\end{center}
\caption{Results Vector Model(without idf-weighting)}
\end{table}

\pagebreak
\section{Two examples which have different best result ( with/without idf)}
\subsection{Case 5}
Pasienten er en 64 år gammel kvinne som tidligere er stort sett frisk. De siste 5 månedene har hun merket følelse av ufullstendig tømming når hun har avføring. Ved flere anledninger har det vært spor av friskt blod i avføringen. Selv tilskriver hun dette hemorroider som hun har hatt før, og hun har ventet på at det skulle gå over.

Hun har vært hos sin egen lege som ikke finner noe galt ved vanlig undersøkelse. Ved rektaleksplorasjon (kjenne i endetarmsåpningen med finger) kjenner legen så vidt kanten av en uregelmessighet i tarmveggen. Legen ser spor etter gamle ytre hemorroider men ingen sannsynlig blødningskilde nå. Prøver på blod i avføringen er positive. Blodprøver viser en lett blodmangel (hemoglobin 10.8; normalt for kjønn og alder er >12). Øvrige blodprøver var normale. Pasienten blir henvist til colonoscopi (endoskopisk undersøkelse av tykktarmen).

\subsubsection{Results (Best match)}
\begin{description}
\item{\textbf{Without idf: }}T5.5 Depresjoner(Depressions)
\item{\textbf{With idf: }}T12.10 Anorektale Forstyrrelser(Anorectal disorders)
\end{description}

Here we see clearly that T5.5 has nothing to do with this old woman’s rectal issues. T12.10, which is the best match using idf weighting, on the other hand, is a good match for her symptoms. When using plain term frequency as the weighting scheme the T12.10 chapter isn’t even on the top 10 list. In this case, we therefore see a clear performance gain from the idf weighting here.

\subsection{Case 6}
Ved første konsultasjon ble det funnet forstørrede tonsiller med hvitlig belegg og forstørrede glandler på begge sider av halsen. Det ble tatt halsprøve til strept.test som var positiv. Det fremkom at kjæresten nylig hadde gjennomgått streptokokktonsilitt. Det ble startet behandling med peroral penicillin (Apocillin) i vanlig dosering.

Pasienten kommer til ny konsultasjon etter manglende effekt av behandlingen med penicillin. Han ble verre, fikk mer svelgbesvær, fikk ikke i seg fast føde, og hadde også besværligheter med å få i seg væske. Samtidig var han vedvarende slapp og i dårlig allmenntilstand.

\subsubsection{Results (Best match)}
\begin{description}
\item{\textbf{Without idf: }}T2.2 Pharmacological treatment of common cancers.
\item{\textbf{With idf: }}T1.3 Mononucleosis.
\end{description}
The results without idf suggest a therapy chapter about cancer treatment which clearly has not much to do with the patients neck problems. However with idf we see that Mononucleosis is a much better suggestion since it is a sub-chapter of infectious diseases. The result with idf is more relevant for det case description than without idf.